
\documentclass [11pt]{report}

\usepackage{fancyhdr}
\usepackage [french]{babel}

\usepackage[utf8]{inputenc}
\usepackage[T1]{fontenc}
\usepackage{textcomp}
\usepackage{graphicx}
\usepackage{titlepic}
\usepackage{boxedminipage}
\usepackage{listings}
\usepackage{minitoc}

\pagestyle{fancy}



\title{
	\includegraphics[scale=0.43]{images/entete.jpg}
	 \\\vspace{20mm}
	\textbf{\Huge \itshape Cahier des charges }
	}




\author{ \\\vspace{2mm}
	Adrien Louge \\\vspace{2mm}
	Florent Youinou\\\vspace{2mm}
	Mathilde Laplaze\\\vspace{2mm}
	Thibault Gdalia\\\vspace{30mm}
	}



\date{17 janvier 2014}


\begin{document}

\renewcommand{\baselinestretch}{0.001}
\maketitle
\tableofcontents

\newpage
\chapter {Pr\'esentations}

	\section{ L'\'equipe }

		*nom du groupe* est  un groupe constitué de quatre personnes : Adrien Louge alias "TiNyDGZz", le chef de ce projet, Mathilde Laplaze alias "Mattou", la seule fille du groupe, Florent Youinou alias "T4ze" et Thibault Gdalia alias "Skeat". Retenez bien ces pseudos car dans le reste de ce Cahier des Charges nous n'utiliserons plus nos noms et pr\'enoms. Nous sommes r\'epartie sur deux classes : la SUPB2 et la SUPD2.  \'Etant donné que le groupe compte deux redoublants, TiNyDGZz et Skeat, qui ont d\'eja eu cette configuration de groupe l'ann\'ee derni\`ere, nous savons qu'il n'est pas impossible de travailler entre \'el\`eves de classes diff\'erentes. Le groupe a \'et\'e form\'e en fonction des affinit\'es, ainsi que des comp\'etences de chacun que nous avons pu observer durant le premier semestre. Pour une meilleure ambiance dans l'\'equipe, et pour avoir un bon rythme de travail, le groupe s'est constitué de membres qui ont la volont\'e de travailler. 
	
	Nous sommes partis sur un jeu en 2D bas\'e sur un principe tr\`es connu, que l'on retrouve dans Jetpack ou encore Badland, tout en y apportant notre touche personnel.
	
	\newpage

	\section { Les Membres }
		\subsection {Mathilde "Mattou" Laplaze}
			%le texte ici %
	
		
		\newpage

		\subsection {Adrien "TiNyDGZz" Louge}
			%le texte ici %
	
		\newpage

		\subsection {Florent "T4ze" Youinou}
		T4ze, c’est moi. \'Etant fils d’ingénieur en informatique, ma passion me vient de mon père et contrairement aux enfants de mon âge, je n’ai jamais été un grand fan des jeux vidéo. Ainsi pendant que les autres s’amusaient sur leur console, moi je bidouillais sur mon ordinateur. J’ai donc commencé tôt à coder. Connaissant mes ambitions je me suis très vite investit dans les études qui me permettrait d’y accéder. Le problème c’est que du coup pour moi, les autres matières étaient totalement inutile et me faisait perdre mon temps. J’ai donc suivi passé un bac S SI (Science de l’Ingénieur) avec la spé ISN. J’adore apprendre et écouter les suggestions construites de personnes extérieurs.\\
\indent Mais l’idée de vivre accroché à un siège devant un écran ne donne pas envie, heureusement j’ai d’autres passions pour me changer les idées. En effet je fais beaucoup de sport, je sors souvent et ma plus grande source d’inspiration pour coder me vient des films que je regarde.

	
		\newpage
		
		\subsection {Thibault "Skeat" Gdalia}

			Moi c'est Skeat, je suis en SUP a EPITA pour la deuxi\`eme fois (oui j'ai redoubl\'e),  c'est donc mon deuxi\`eme projet, l'ann\'ee derni\`ere au sien du groupe DAMNIT. Cette ann\'ee je suis toujours motiv\'e pour travailler en groupe, et accro\^itre mes comp\'etences en C\# . 
			\\ \indent  Avant de rentrer \'a EPITA, j'ai fait une terminal S Sciences de l'Ing\'enieur, depuis que je suis en prem\'iere je code, j'ai commenc\'e par le Visual BAsic en cours au lyc\'ee puis je me suis tourn\'e vers le C++. Je n'ai jamais \'et\'e un grand fan de jeu vid\'eo, j'ai toujours pr\'ef\'er\'e faire du sport ou sortir faire la f\^ete (donc rien a voir avec un GEEK). L'ann\'ee derni\`ere je n'ai pas valid\'e mon ann\'ee car je suis arriv\'e avec trop de lacune, mais cette ann\'ee va \^etre une autre histoire, c'est pour cela que j'ai choisi de faire parti de ce groupe qui \'a l'air tr\`es motiv\'e. De plus je connais tr\`es bien TiNyDGZz avec qui j'ai travaill\'e l'ann\'ee derni\`ere.
	
	\newpage\

\chapter{Principe du jeu}
Le personnage doit parcourir les différentes maps du jeu, en évitant les nombreux obstacle se trouvant sur son chemin. Afin que le jeu est un peu plus d'interet pour le joueur, les maps devront être debloquer les unes a la suite des autres, et deviendront de plus en plus dur, avec de nouveau obstacle.\\
le joueur ne pourra pas déplacer son personnage comme il le souhaite il pourra juste lui faire faire un bond en appuyant sur espace (jusque la rien de bien nouveau), en plus de cela il devra manger les bonbons qui se trouvent sur son chemin afin de faire remonter sa barre d'energie. A chaque fois que le joueur touche un obstacle il est ralenti, a chaque bond effectué le joueur consomme de l'énergie, pour regagner de l'énergie le joueur a deux possibilités laisser l'oiseaux planer ou manger un bonbons. La partie s'arrête lorsque le joueur sort de l'écran (par la droite), cela arrive si le joueur touche trop d'obstacle, ou la partie est finie quand le joueur 


\chapter {Graphismes}
	\section {Logo}
 		% le texte ici %
\chapter {Moteur Physique}
	% le texte ici %
\chapter {Site Internet}
	% le texte ici %
\chapter {Reseaux}
	% le texte ici %
\chapter{Répartitions des tâches}
	\subsection{Les Rôles}
		\begin{tabular}{| l |*{4} {r|}}
		\hline
		Themes & Mathilde & Adrien & Florent & Thibault \\
		\hline
		Son & X & & & \\
		\hline
		Reseau & & & X & X \\
		\hline
		site Internet & & & X & \\
		\hline
		Graphisme & X & X & & \\
		\hline
		Moteur Physique & & X & X & \\
		\hline
		\end{tabular}

	\subsection{Planning des soutenances}
		\begin{tabular}{| l | * {3}{r |}}
		\hline
		 & Soutenance 1 & Soutenance 2 & Soutenance 3 \\
		\hline
		Moteur Physique & & & \\
		\hline
		 Graphisme & & & \\
		\hline
		Réseaux & & & \\
		\hline
		site internet & 50\% & 100\%  & 100\%  \\
	           \hline
		\end{tabular}
\chapter {Conclusion}
	% le texte ici %
\end {document}